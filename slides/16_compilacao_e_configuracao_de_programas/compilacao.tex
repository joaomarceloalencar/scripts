\documentclass{beamer}

\usepackage[portuguese]{babel}
\usepackage[utf8]{inputenc}
\usepackage{minted}

\title{Compilação e Configuração de Programas}
\author[João Marcelo Uchôa de Alencar]{João Marcelo Uchôa de Alencar}
\institute{Universidade Federal do Ceará - Quixadá}

% http://www.adp-gmbh.ch/cpp/gcc/create_lib.html 
% http://www3.ntu.edu.sg/home/ehchua/programming/cpp/gcc_make.html#zz-2.

\begin{document}
   \begin{frame}
      \titlepage
   \end{frame}

   \begin{frame}
      \frametitle{Introdução}
      Você não precisa ser programador em C/C++ para administrar um sistema Linux
      \begin{itemize}
         \item Entretanto, boa parte dos serviços é escrita em C/C++
	 \item O \textit{kernel} é em C
	 \item Mesmo sem entender a linguagem, é útil saber compilar um programa em C/C++ 
      \end{itemize}
   \end{frame}

   \begin{frame}
      \frametitle{Ambiente de Compilação GNU}
      \begin{itemize}
         \item GCC: coleção de compiladores
	 \item Make: ferramenta para automatizar a compilação
	 \item Binutils: \textit{linker} e \textit{assembler}
	 \item Autotools: sistema de construção
      \end{itemize}
      \$ sudo apt-get install g++ gcc make autoconf
   \end{frame}

   
   \begin{frame}[fragile]
      \frametitle{Um Programa Simples em C/C++}
      \begin{minted}{c}
      #include <stdio.h>
       
      int main() {
         printf("Redes de Computadores!\n");
         return 0;
      }
      \end{minted}
      \begin{minted}{c}
      // hello.cpp
      #include <iostream>
      using namespace std;
       
      int main() {
         cout << "Hello, world!" << endl;
	 return 0;
       }
      \end{minted}
      \$ gcc -o hello hello.c \\
      \$ g++ -o hello hello.cpp
\end{frame}

   \begin{frame}[fragile]
      \frametitle{Criando Uma Biblioteca Compartilhada}
      \begin{minted}{c}
      // media.c
      double media(double a, double b) {
         return (a+b) / 2;
      }
      // media.h
      extern double media(double, double);
      // main.cpp
      #include <stdio.h>
      #include "media.h"
      int main(int argc, char* argv[]) {
         double v1, v2, m;
         v1 = 5.2;
         v2 = 7.9;
         m = media(v1, v2);
         printf("%3.2f %3.2f %3.2f\n", v1, v2, m);
         return 0;
      }
      \end{minted}
\end{frame}

   \begin{frame}[fragile]
      \frametitle{Criando uma Biblioteca Compartilhada}
      Coloque os arquivos \textit{media.c} e \textit{media.h} em um diretório chamado \textit{media} \\
      \begin{minted}{bash}
      :~/media$ gcc -c -Wall -Werror -fpic media.c
      :~/media$ gcc -shared -o libmedia.so media.o
      :~/$ gcc -Imedia/ -Lmedia/ main.c -o main -lmedia
      :~/$ ./main 
      ./main: error while loading shared libraries: 
      libmedia.so: cannot open shared object file: 
      No such file or directory
      :~/$ LD_LIBRARY_PATH=$PWD/media ./main
      A média de 5.20 e 7.90 é 6.55
      \end{minted}
      Ao executar, o programa procura as bibliotecas compartilhadas nos diretórios listados na variável de ambiente \textit{LD\_LIBRARY\_PATH}
\end{frame}

   \begin{frame}[fragile]
      \frametitle{Criando uma Biblioteca Compartilhada}
      Você pode facilitar a compilação configurando as variáves \textit{LIBRARY\_PATH} e \textit{CPATH}
      \begin{minted}{bash}
      :~/$ export CPATH=$PWD/media
      :~/$ export LIBRARY_PATH=$PWD/media
      :~/$ gcc main.c -o main -lmedia
      :~/$ export LD_LIBRARY_PATH=$PWD/media
      \end{minted}
\end{frame}

   \begin{frame}
      \frametitle{Exercícios - Parte I}
      Crie um \textit{script} chamado \textit{compilaeinstala.sh} que:
      \begin{itemize}
         \item Compila os arquivos .c e .h do exemplo anterior
	 \item Cria os diretórios \textit{bin}, \textit{lib} e \textit{include} na pasta raiz do usuário
	 \item Move o binário gerado para a pasta \textit{bin}, a biblioteca para a pasta \textit{lib} e o cabeçalho para \textit{include}
	 \item Atualiza o arquivo \textit{.profile} ou \textit{.bashrc} do usuário para que possa executar o binário como qualquer outro comando
      \end{itemize}
   \end{frame}

   \begin{frame}[fragile]
      \frametitle{A Ferramenta \textit{make}}
      \begin{itemize}
         \item Automatiza a compilação de código
	 \item É baseado no conceito de alvos e pré-requisitos
	 \item As etapas da compilação são armazenadas em um arquivo \textit{makefile}
	 \item Foi criada para compilar, mas nada impede que seja usada para automatizar outras tarefas
      \end{itemize}
      \begin{minted}{bash}
      alvo1: alvo0
             comando0
	     comando1
	     ...
      \end{minted}
      \begin{itemize}
         \item Cada comando é executado dentro de um processo \textit{shell} diferente. Não podemos considerar como um \textit{script} normal
      \end{itemize}
\end{frame}


   \begin{frame}[fragile]
   \scriptsize
   \begin{minted}{bash}
INCLUDE_DIR=media/
LIBRARY_DIR=media/
install:	
	[ -d ~/bin ] || mkdir ~/bin
	cp main ~/bin/
	[ -d ~/lib ] || mkdir ~/lib
	cp media/libmedia.so ~/lib/
	[ -d ~/include ] || mkdir ~/include
	cp media/media.h ~/include/
	echo "PATH=$$PATH:~/bin" >> ~/.profile
	echo "LD_LIBRARY_PATH=$$LD_LIBRARY_PATH:~/lib" >> ~/.profile
	echo "export PATH LD_LIBRARY_PATH" >> ~/.profile
uninstall:
	rm ~/bin/main
	rm ~/lib/libmedia.so
	rm ~/include/media.h

all: main
main: media.o libmedia.so
	gcc -I$(INCLUDE_DIR) -L$(LIBRARY_DIR) main.c -o main -lmedia
libmedia.so: media.o
	cd media && gcc -shared -o libmedia.so media.o
media.o:
	cd media && gcc -c -Wall -Werror -fpic media.c
clean:
	rm main
	rm media/media.o
	rm media/libmedia.so
   \end{minted}
\end{frame}

   \begin{frame}
      \frametitle{Exemplo de Compilação - \textit{pdsh}}
      Nós vamos instalar o \textit{psdh}
      \begin{itemize}
         \item É um programa que facilita a execução de comandos e \textit{scripts} em várias máquinas diferentes
	 \item Durante a compilação, um \textit{makefile} é criado de acordo com as bibliotecas presente no sistema
	 \item Você pode informar opções ao \textit{script} que cria o \textit{makefile}, tornando bem mais fácil a compilação
      \end{itemize}
   \end{frame}
   
   \begin{frame}[fragile]
      \frametitle{Passos para Instalação}
      \begin{minted}{bash}
      $ mkdir $HOME/tools
      $ wget goo.gl/RBu3U8
      $ mv RBu3U8 pdsh-2.29.tar.bz2
      $ tar -xjvf pdsh-2.29.tar.bz2
      $ cd pdsh-2.29
      $ ./configure --prefix=$HOME/tools --with-ssh
      $ make
      $ make check
      $ make install
      $ cat<<EOF >> $HOME/.profile
      PATH=$PATH:$HOME/tools/bin
      MANPATH=$MANPATH:$HOME/tools/man
      LD_LIBRARY_PATH=$LD_LIBRARY_PATH:$HOME/tools/lib
      export PATH MANPATH LD_LIBRARY_PATH
      EOF
      \end{minted}
\end{frame}

   \begin{frame}[fragile]
      \frametitle{Utilizando o \textit{pdsh}}
      \begin{minted}{bash}
      $ ssh-keygen 
      $ ssh-copy-id alunoufc@192.168.0.2
      $ ssh-copy-id alunoufc@192.168.0.3
      $ pdsh -w alunoufc@192.168.0.2-3 date
      192.168.0.2: Tue Oct 25 11:57:34 BRT 2016
      192.168.0.3: Tue Oct 25 11:58:27 BRT 2016
      \end{minted}
      Para executar um \textit{script} remotamente, o mesmo deve ser copiado antes através do comando \textit{scp}.
\end{frame}

   \begin{frame}
      \frametitle{Exercício - Parte I}
      Instale, na sua máquina, sem usar o usuário \textit{root}, o utilitário \textit{pdsh}. 
   \end{frame}

   \begin{frame}[fragile]
      \frametitle{Exercício - Parte II}
      O comando \textit{uptime} retorna, nós três últimos valores numéricos, a carga média do sistema nos últimos um, cinco e quinze minutos. Faça um \textit{script.sh} chamado \textit{clusterLoad.sh} que execute o comando \textit{uptime} em um conjunto de máquinas através do \textit{pdsh} e retorne uma lista ordenada pela carga no último minuto. O conjunto de máquinas que o \textit{pdsh} vai atuar deve ser passado como parâmetro. 
      \scriptsize
      \begin{minted}{bash}
      $ ./clusterLoad.sh 192.168.0.2 192.168.0.14 192.168.0.100
      192.168.0.2: 0,40
      192.168.0.100: 0.62
      192.168.0.15: 0,77
      \end{minted}
\end{frame}

   \begin{frame}
      \frametitle{Exercício - Parte III}
      Faça um \textit{script} chamado \textit{mapAndReduce.sh} com a seguintes funcionalidades:
      \begin{itemize}
         \item Ele deve receber como primeiro parâmetro o nome de um arquivo, possivelmente muito grande, contendo texto.
	 \item Os parâmetros seguintes são endereços de máquinas, nas quais a chave para o \textit{pdsh} já se encontram devidamente instaladas. Também estão copiados nessas máquinas o \textit{script} para contar palavras, seja na versão \textit{awk} ou \textit{shell}
	 \item O \textit{script} deve repartir o texto em $N$ arquivos, onde $N$ é o número de máquinas informado. Cada arquivo deve conter aproxidamente o mesmo número de linha. 
	 \item Através do \textit{scp}, cada arquivo temporário deve ser enviado para uma máquina. Através do \textit{pdsh}, deve ser iniciado um \textit{script} em cada máquina para contar as palavras da sua versão local.
	 \item O \textit{script} deve consolidar o resultado na contagem total de cada palavra.
      \end{itemize}
   \end{frame}

\end{document}
