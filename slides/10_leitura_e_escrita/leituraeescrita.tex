\documentclass{beamer}

\usepackage[portuguese]{babel}
\usepackage[utf8]{inputenc}
\usepackage{minted}

\title{Leitura e Escrita}
\author[João Marcelo Uchôa de Alencar]{João Marcelo Uchôa de Alencar}
\institute{Universidade Federal do Ceará - Quixadá}

\begin{document}
   \begin{frame}
      \titlepage
   \end{frame}

   \begin{frame}
      \frametitle{Comando \textit{tput}}
      Permite controlar características da do texto que vai ser impresso e a posição do cursor. \\
      \begin{itemize}
         \item Uso básico: \textbf{tput cup} (\textit{cup} significa \textit{cursor position})
	      \item \textbf{tput cup lin col}: colocar o curso na linha \textit{lin} e coluna \textit{col}
      \end{itemize}
      Só dá para perceber mesmo efeito dentro de \textit{scripts}.
   \end{frame}

   \begin{frame}[fragile]
   \begin{minted}{bash}
   #!/bin/bash
   echo "Escrita Normal."
   tput setaf 4

   tput bold
   echo "Negrito."

   tput smso
   echo "Reverso."

   tput smul
   echo "Sublinhado."

   linhas=`tput lines`
   colunas=`tput cols`
   echo "Linhas: $linhas, Colunas: $colunas"

   tput cup `expr $linhas / 2` `expr $colunas / 2`
   echo "Centro."
   \end{minted}
\end{frame}

%   \begin{frame}[fragile]
%      \frametitle{Exemplo}
%      Vamos escrever um \textit{script} chamado \textit{formataTexto.sh} que recebe quatro parâmetros: 
%      \begin{enumerate} 	     
%         \item \textbf{Primeiro Parâmetro}: um das três opções \textit{sublinhado, negrito ou reverso}. 
%         \item \textbf{Segundo Parâmetro}: um número que representa a cor de acordo com o comando \textit{tput}. 
%         \item \textbf{Terceiro Parâmetro}: dois números separados por vírgula. 
%         \item \textbf{Quarto Parâmetro}: um texto a ser impresso. 
%      \end{enumerate}
%      Por exemplo: 
%      \begin{minted}{bash}
%$ ./formataTexto.sh negrito 1 20,20 "Scripts"
%      \end{minted}
%      Imprime o texto na posição 20,20, na cor vermelha e em negrito.
%\end{frame}

   \begin{frame}[fragile]
      \frametitle{Comando \textit{read}}
      \begin{minted}{bash}
      $ read [var1] [var2] [var3] ... 
      \end{minted}
      Onde \textit{var1, var2, var3} são variáveis separadas por um ou mais espaços em branco ou TAB e os valores informados também deveram conter um ou mais espaços entre si. 
      \begin{minted}{bash}
      $ read -p "Informe seu nome:" nome 
      \end{minted}
      O parâmetro $-p$ permite adicionar uma frase antes da leitura.
\end{frame}

   \begin{frame}[fragile]
   \begin{minted}{bash}
   $ cat numeros.txt 
   1 2 3
   4 5 6
   7 8 9
   $ cat numeros.txt | while read a b c
   > do d=$(($a + $b + $c)); echo $d; done;
   6
   15
   24
   \end{minted}
\end{frame}

   \begin{frame}[fragile]
      \frametitle{Comando \textit{printf}}
      \begin{minted}{bash} 
   $ printf formato [argumento ...]         
      \end{minted}
      \begin{itemize}
         \item formato: é uma cadeiaa de caracteres que contém três tipos de objetos: caracteres simples, caracteres de formato e sequências de escape.
	      \item é a cadeia a ser impressa
      \end{itemize}
      Cada um dos caracteres utilizados para especificação de formato é precedido pelo caractere \% e logo a seguir vem a especificação de formato.
\end{frame}

%   \begin{frame}
%      \frametitle{Atividade}
%      Coloque o \textit{script} resultante em \textit{atividades/atividades08}. \\
%      Observe o arquivo \textit{/home/compartilhado/calculadora.sh} no servidor. Altere o programa para atender os seguintes critérios:
%      \begin{enumerate}
%         \item Garantir que todas as operações aritméticas básicas (adição, multiplicação, subtração e divisão) estejam funcionando para os inteiros.
%	      \item Garantir que todas as operações aritméticas básicas funcionem para números de ponto flutuante (use o ponto . para separar a parte inteira da parte decimal). 
%      \end{enumerate}
%   \end{frame}

\end{document}
