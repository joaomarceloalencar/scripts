\documentclass{beamer}

\usepackage[portuguese]{babel}
\usepackage[utf8]{inputenc}
\usepackage{minted}

\title{Variáveis e Vetores}
\author[João Marcelo Uchôa de Alencar]{João Marcelo Uchôa de Alencar}
\institute{Universidade Federal do Ceará - Quixadá}

\begin{document}
   \begin{frame}
      \titlepage
   \end{frame}

   \begin{frame}
      \frametitle{O \textit{Subshell}}
      \begin{itemize}
         \item Cada execução de um \textit{script} inicia um novo processo \textit{shell} que vai interpretar as instruções nele contidas. 
         \item As variáveis definidas no \textit{shell} pai não são copiadas automaticamente para o \textit{shell} filho. 
         \item Para copiar as variáveis, precisamos usar o comando \textbf{export}. 
         \begin{itemize}
            \item Qualquer variável não exportada é ignorada pelos \textit{subshells}.
	         \item O \textit{subshell} acessa uma \textbf{cópia} das variáveis.
	         \item Um \textit{subshell} pode exportar uma variável para seus \textit{subshells}.
         \end{itemize}
         \item Uma maneira de retornar um valor de um \textit{subshell} é atribuir o retorno a uma variável.
	 \end{itemize}
   \end{frame}

   \begin{frame}
      \frametitle{O Comando \textit{source}}
      \begin{itemize}
         \item Através do \textit{source}, você pode executar os comandos de um arquivo no próprio \textit{shell} atual. 
         \item Dessa forma, suas variáveis são atualizadas.
         \item Os arquivos \textit{.profile} ou \textit{.bashrc} são invocados com \textit{source} toda vez que você faz \textit{login}.  
      \end{itemize}
   \end{frame}

   \begin{frame}
      \frametitle{Principais Variáveis do Sistema}
      \begin{table}
         \begin{tabular}{c|c}
         HOME  & Caminho do seu diretório\\
         \hline
         PATH  & Caminhos que guardam programas\\
         \hline 
         PWD  & Diretório corrente\\
         \hline
         LOGNAME & Seu nome de usuário \\
         \hline 
         PS1  & Caracteres do \textit{prompt} primário\\
         \hline 
         IFS  & Separador padrão\\
         \hline 
         TERM  & Tipo de terminal \\
         \hline 
	      PROMPT\_COMMAND & Executado ao final de cada comando \\
         \end{tabular}
      \end{table}
   \end{frame}

   \begin{frame}
      \frametitle{Exemplo}
      \begin{itemize}
         \item Vamos criar uma variável no arquivo \textit{.bashrc} chamada \textbf{LOGIN\_DATE}. Ela deve armazenar a data que fazemos \textit{login} no sistema.
	      \item Vamos criar um diretório \textbf{bin} na sua pasta de usuário. Dentro dele, um \textit{script} chamado \textit{showLoginDate} (sem a extensão .sh) deve exibir na tela o conteúdo da variável \textbf{LOGIN\_DATE}. Esse diretório deve ser inserido na variável \textbf{PATH}.
      \end{itemize}
   \end{frame}

   \begin{frame}
      \frametitle{Usando as \{ \} para Controlar Variáveis}
      \begin{itemize}
         \item \$\{!variavel\}: indireção.
	      \item \$\{!padrao$*$\}: nome de variáveis de acordo com o padrão.
	      \item \$\{variavel:-default\}: valor \textit{default}.
	      \item \$\{variavel:=default\}: atualiza a variável com valor \textit{default}.
         \item \$\{variavel\slash de\slash para\} substitui uma ocorrência do padrão.
         \item \$\{variavel\slash\slash de\slash para\} substitui todas as ocorrências.
      \end{itemize}
   \end{frame}
 
   \begin{frame}
      \frametitle{Usando as \{ \} para Gerar \textit{Strings}}
      \begin{itemize}
         \item \{lista\}
	      \item \{inicio..fim\}
	      \item prefixo\{lista\} ou prefixo\{inicio..fim\}
	      \item \{lista\}sufixo ou \{inicio..fim\}sufixo
	      \item prefixo\{lista\}sufixo ou prefixo\{inicio..fim\}sufixo
	      \item \{inicio..fim..incr\}
      \end{itemize}
   \end{frame}

   \begin{frame}
      \frametitle{Vetores ou \textit{Arrays}}
      \begin{itemize}
         \item vet=(elemento0 elemento1 elemento2 ... elementoN)
	      \item vet[n]=valor
	      \item declare -a vet (sem alterar caso exista)
	      \item declare -A vet (vetor associativo)
	      \item read -a vetor
      \end{itemize}
   \end{frame}

   \begin{frame}
      \frametitle{Acessando Vetores}
      \begin{itemize}
         \item \$\{vet[0]\}
         \item \$\{!vet[@]\}
	 \item \$\{\#vet[@]\}
	 \item \$\{vet[@]:0:2\}
      \end{itemize}
   \end{frame}
   
   \begin{frame}
      \frametitle{Exemplo}
      \begin{itemize}
        \item Façamos um \textit{script} chamado \textit{contadorVetor.sh} 
	      \item O usuário deve informar, sem utilizar parâmetros, um número por vez.
	      \item O número deve ser armazenado em um vetor. 
	      \item Quando o usuário informar o caractere \textit{q} o \textit{script} deve informar quantos números foram inseridos no vetor e terminar. Devemos usar vetores!
      \end{itemize}
   \end{frame}

%   \begin{frame}[fragile]
%      \frametitle{Atividade}
%      Faça um \textit{script} chamado \textit{contaPalavras.sh} na pasta \textit{atividades/atividades08} que pergunte ao usuário o nome de um arquivo de texto e para cada palavra do arquivo diga quantas vezes ela aparece no texto. \\
%      \begin{minted}{bash}
%      $ cat arquivo.txt
%      a casa que vivo é boa.
%      boa casa é.
%      $ ./contaPalavras.sh
%      Informe o arquivo: arquivo.txt
%      casa: 2
%      boa:  2
%      é:    2
%      a:    1
%      que:  1
%      vivo: 1
%      \end{minted}
%\end{frame}

\end{document}
