\documentclass{beamer}
\usepackage[portuguese]{babel}
\usepackage[utf8]{inputenc}
\usepackage{minted}

\setbeamertemplate{footline}[frame number]

\title{Comandos Básicos}
\author[João Marcelo Uchôa de Alencar]{João Marcelo Uchôa de Alencar}
\institute{Universidade Federal do Ceará - Quixadá}

\begin{document}
   \begin{frame} 
      \titlepage
   \end{frame}

   \begin{frame}
      \frametitle{Revisão do Básico do Sistema de Arquivos}
      \begin{columns}
      \begin{column}{0.4\textwidth}
      / \\
      \hspace{0.5cm}/etc  \\
      \hspace{1.0cm}/etc/passwd \\
      \hspace{1.0cm}/etc/rc.d \\
      \hspace{1.0cm} ... \\
      \hspace{0.5cm}/home \\ 
      \hspace{1.0cm}/home/joaomarcelo \\ 
      \hspace{1.0cm}/home/aluno01 \\ 
      \hspace{1.0cm} ... \\
      \hspace{0.5cm}/usr \\ 
      \hspace{1.0cm}/usr/local \\ 
      \hspace{1.0cm}/usr/lib \\ 
      \hspace{1.0cm} ... \\
      \end{column}
      \begin{column}{0.6\textwidth}
      \scriptsize
      \begin{itemize}
         \item A partição de inicialização é montada na raiz (usando o símbolo /).
	 \item Partindo da raiz, outros diretórios são organizados de acordo com cada distribuição, mas há um padrão geral:
	 \begin{itemize}
           \scriptsize
	    \item /etc: arquivos de configuração específicos do computador.
	    \item /boot: imagem do \textit{kernel}.
	    \item /dev: dispositivos.
	    \item /home: diretórios dos usuários.
	    \item /lib: bibliotecas básicas.
	    \item /bin: executáveis acessíveis por todos os usuários.
	    \item ...
	 \end{itemize}
	 \item \url{https://pt.wikipedia.org/wiki/Filesystem_Hierarchy_Standard}
      \end{itemize}
      \end{column}
      \end{columns}
   \end{frame}

\begin{frame}[fragile]
   \frametitle{Comandos no Linux}
   Formato geral: 
   \begin{minted}{bash}
   $ comando [-opções...] [argumentos...] 
   \end{minted}
   Para obter ajuda:
   \begin{minted}{bash}
   $ comando --help (ou -help)
   $ man comando
   \end{minted}
   \begin{itemize}
      \item Pressionar a tecla \textit{seta para cima} repete os últimos comandos digitados.
      \item Você pode apenas digitar o início do comando e apertar TAB, o \textit{shell} completa o comando (ou caminho do diretório).
      \item Caso o \textit{shell} não complete, TAB duas vezes mostra as opções disponíveis.
      \item CTRL+R permite buscar nos últimos comandos digitados.
   \end{itemize}
\end{frame}

\begin{frame}[fragile]
   \frametitle{pwd - Informa nome do diretório corrente}
   \begin{minted}{bash}
   $ pwd 
   /home/joaomarcelo 
   \end{minted}
   \begin{itemize}
      \item Retorna o diretório atual.
      \item Não parece muito útil a primeira vista.
      \item Em \textit{scripts} vai permitir que verifiquemos se estamos no diretório correto.
   \end{itemize}
\end{frame}

\begin{frame}[fragile]
   \frametitle{cd - Navegando entre diretórios}
   \begin{minted}{bash}
   $ cd /usr/local 
   $ pwd
   /usr/local 
   \end{minted}
   Os seguintes atalhos podem ser usados tanto no \textit{cd} quanto em outros comandos:
   \begin{table}
      \begin{tabular}{ c | c }
         . (ponto) & Diretório atual \\
         \hline 
         .. (dois pontos) & Diretório acima na hierarquia \\
         \hline
         \textasciitilde (acento til) & Diretório \textit{home} do usuário \\
         \hline
         / (barra) & Diretório raiz \\
         \hline 
         - (hífen) & Último diretório visitado \\
      \end{tabular}
   \end{table}
\end{frame}

\begin{frame}[fragile]
   \frametitle{ls - Lista Arquivos}
   \begin{minted}{bash}
   $ ls [opções] [arquivo/diretório] 
   \end{minted}
   \begin{table}
      \begin{tabular}{ c | c }
      -l & Lista detalhada \\
      \hline 
      -a & Lista arquivos ocultos \\
      \hline
      -r & Ordem reversa \\
      \hline
      -h & Formato legível \\
      \hline 
      -R & Listar subdiretórios \\
      \end{tabular}
   \end{table}
\end{frame}

\begin{frame}[fragile]
   \frametitle{cp - Cópia de arquivos e diretórios}
   \begin{minted}{bash}
   $ cp [opções] <origem> <destino>
   \end{minted}
   \begin{table}
      \begin{tabular}{ c | c }
      -i & Modo interativo \\
      \hline 
      -v & Exibe cada arquivo copiado \\
      \hline
      -r & Copia diretórios e subdiretórios \\
      \hline
      -p & Tenta preservar o máximo de atributos \\
      \end{tabular}
   \end{table}
   Em \textit{scripts}, talvez a opção mais importante seja $-p$.
\end{frame}

\begin{frame}[fragile]
   \frametitle{mv - Move arquivos e diretórios}
   \begin{minted}{bash}
   $ mv [opções] <origem> <destino> 
   \end{minted}
   \begin{itemize}
      \item Mover é uma maneira de renomear arquivos!
      \item Mover costuma ser mais lento, principalmente se for entre partições diferentes.
      \item Pode ser aplicado tanto para arquivos quanto diretórios.
      \item A opção $-i$ pergunta antes de sobreescrever arquivos existentes. Não confie que ela está ativa!
   \end{itemize}
\end{frame}

\begin{frame}[fragile]
   \frametitle{ln - Ligações entre arquivos}
   \begin{minted}{bash}
   $ ln [-s] <origem> <ligação>
   \end{minted}
   \begin{itemize}
      \item \textit{Links} são atalhos.
      \item \textit{Hard Link}: o arquivo de ligação tem o mesmo \textbf{inode} do arquivo de origem.
      \item \textit{Soft Link}: o arquivo de ligação contém apenas o caminho do arquivo de origem.
      \item A \textit{origem} pode ser tanto arquivos quanto diretórios.
   \end{itemize}
\end{frame}

\begin{frame}[fragile]
   \frametitle{mkdir - Cria um diretório}
   \begin{minted}{bash}
   $ mkdir [opções] <nomedodiretorio>
   \end{minted}
   \begin{itemize}
      \item Você pode usar todas as abreviações do comando \textit{cd}.
      \item Subdiretórios não são criados automaticamente.
   \end{itemize}
\end{frame}

\begin{frame}[fragile]
   \frametitle{rmdir - Remove diretório}
   \begin{minted}{bash}
   $ rmdir <nomedodiretorio>
   \end{minted}
   \begin{itemize}
      \item Para usar esse comando, o diretório tem que estar vazio!
      \item Pode parecer uma limitação, mas é bom usar em um \textit{script} quando deseja ter certeza que o diretório excluído está vazio.
   \end{itemize}
\end{frame}

\begin{frame}[fragile]
   \frametitle{rm - Deleta arquivos e diretórios}
   \begin{minted}{bash}
   $ rm [opções] <arquivos>
   \end{minted}
   \begin{table}
      \begin{tabular}{ c | c }
         -f & Modo força bruta \\
         \hline 
         -I & Modo confirmação \\
         \hline
         -r & Remove diretórios \\
      \end{tabular}
   \end{table}
   A opção $-r$ remove mesmo se o diretório contenha arquivos e subdiretórios.
\end{frame}

\begin{frame}[fragile]
   \frametitle{file - Indica o tipo de arquivo}
   \begin{minted}{bash}
   $ file <arquivo>
   \end{minted}
   \begin{itemize}
      \item Extensões de arquivos não tem significado preciso no Linux.
      \item O \textit{file} abre a porção inicial do arquivo e extrai informações sobre sua natureza.
   \end{itemize}
\end{frame}

\begin{frame}[fragile]
   \frametitle{basename e dirname - Retornam nome do arquivo e nome do diretório}
   \begin{minted}{bash}
   $ basename <arquivo> 
   $ dirname <diretório>
   \end{minted}
   Novamente, não são muito úteis fora de \textit{scripts}.
\end{frame}

\begin{frame}[fragile]
   \frametitle{grep - Pesquisa arquivos por conteúdo}
   \scriptsize
   \begin{minted}{bash}
$ grep [-opções] [expressão] [arquivo1] [arquivo2] [arquivo3] ...
   \end{minted}
   \normalsize
   \begin{table}
      \begin{tabular}{ c | c }
         \hline
         -c & Apenas informar quantas linhas contém a expressão \\
         \hline 
         -i & Não diferenciar maiúsculas e minúsculas \\
         \hline
         -l & Apenas informar quais dos arquivos contém a expressão \\
         \hline
         -v & Busca reversa \\
         \hline
         -n & Exibe o número da linha \\
         \hline
      \end{tabular}
   \end{table}
   A \textit{expressão} é uma E.R. Por enquanto, vamos considerar como uma simples cadeia de caracteres.
\end{frame}

\begin{frame}[fragile]
   \frametitle{find - Procurar arquivos}
   \begin{minted}{bash}
   $ find [caminho] [expressão] [ação]
   \end{minted}
   \textbf{caminho :} diretório a partir do qual a busca começa. \\
   \begin{columns}
   \begin{column}{0.5\textwidth}
   \textbf{expressão :} \\ 
      -name \textit{nome} \\
      -user \textit{usuário} \\
      -group \textit{grupo} \\
      -type \textit{tipo} \\
      -size $\pm$\textit{tamanho} \\
      -atime $\pm$\textit{dias} \\
      -ctime $\pm$\textit{dias} \\
      -mtime $\pm$\textit{dias} \\
   \end{column}
   \begin{column}{0.5\textwidth}
   \textbf{ação :} \\
      -print \\
      -exec \textit{comando} \{\} $\backslash$; \\
      -ok \textit{comando} \{\} $\backslash$; \\ 
      -printf \textit{formato}
   \end{column}
   \end{columns}
\end{frame}

\begin{frame}[fragile]
   \frametitle{cat - Exibe conteúdo de arquivos}
   \begin{minted}{bash}
   $ cat [-opções] [arquivo]
   \end{minted}
   \begin{table}
      \begin{tabular}{ c | c }
         -v & Mostra caracteres sem representação na tela \\
         \hline 
         -e & Mostra a alimentação de linha \\
         \hline
         -t & Mostra os TAB \\
         \hline
         -n & Enumera as linhas \\
      \end{tabular}
   \end{table}
\end{frame}

\begin{frame}[fragile]
   \frametitle{wc - Contar}
   \begin{minted}{bash}
   $ wc [-lwc] [arquivo]
   \end{minted}
   \begin{table}
      \begin{tabular}{ c | c }
         -l & Linhas \\
         \hline 
         -w & Palavras \\
         \hline
         -c & Caracteres \\
      \end{tabular}
   \end{table}
   \begin{itemize}
      \item A maneira mais fácil de saber se dois arquivos são diferentes.
      \item Muito útil para saber se \textit{log} contém novas informações.
   \end{itemize}
\end{frame}

\begin{frame}[fragile]
   \frametitle{head - Exibe linhas a partir do início}
   \begin{minted}{bash}
   $ head [-número] [-cn] [arquivo] 
   \end{minted}
   \begin{table}
      \begin{tabular}{ c | c }
         -número & Linhas a partir do início que são mostradas \\
         \hline 
         -c & Bytes a partir do início que são mostrados \\
         \hline
         -n & igual à $-numero$ \\
      \end{tabular}
   \end{table}
\end{frame}

\begin{frame}[fragile]
   \frametitle{tail - Exibe linhas em relação ao final}
   \begin{minted}{bash}
   $ tail [-número] [+número] [-cnf] [arquivo] 
   \end{minted}
   \begin{table}
      \begin{tabular}{ c | c }
         +número & Mostra a partir da linha \textsf{número} até o fim. \\
         \hline 
         -número & Mostra as \textsf{número} linhas do arquivo. \\
         \hline
         -c & Mostra os últimos \textit{bytes} \\
         \hline
         -n & Igual à -número \\
         \hline
         -f & Exibe o final do arquivo a medida que ele cresce. \\
      \end{tabular}
   \end{table}
\end{frame}

\begin{frame}[fragile]
   \frametitle{sort - Ordenar arquivos}
   \begin{minted}{bash}
   $ sort [opções] [arquivo]
   \end{minted}
   \begin{table}
      \begin{tabular}{ c | c }
         -k & Define a chave de ordenação no formato $n_{1}.m_{1},n_{2}.m_{2}$ \\
         \hline 
         -t & Define o caractere de separação \\ 
         \hline
         -m & Intercala arquivos \\
         \hline
         -n & Classificação númerica \\
         \hline
         -r & Classificação inversa \\
      \end{tabular}
   \end{table}
\end{frame}

\begin{frame}[fragile]
   \frametitle{chown - Trocando o Dono do Arquivo}
   \begin{minted}{bash}
   $ chown [-f] [-R] dono arquivo
   \end{minted}
   \begin{table}
      \begin{tabular}{ c | c }
         arquivo & objeto a ter as permissões alteradas \\
         \hline 
         dono & novo dono do arquivo \\ 
         \hline
         -f & não reportar erros \\
         \hline
         -R & diretórios e subdiretórios \\
      \end{tabular}
   \end{table}
\end{frame}

\begin{frame}[fragile]
   \frametitle{chgrp - Trocando o Dono do Arquivo}
   \begin{minted}{bash}
   $ chgrp [-f] [-R] grupo arquivo
   \end{minted}
   \begin{table}
      \begin{tabular}{ c | c }
         arquivo & objeto a ter as permissões alteradas \\
         \hline 
         dono & novo dono do arquivo \\ 
         \hline
         -f & não reportar erros \\
         \hline
         -R & diretórios e subdiretórios \\
      \end{tabular}
   \end{table}
\end{frame}

\begin{frame}
   \frametitle{Tipos de Acesso}
   \begin{table}
      \begin{tabular}{ c | c }
         dono (\textbf{u}) & usuário dono do arquivo \\
         \hline 
         grupo (\textbf{g}) & o grupo a que pertence o arquivo \\ 
         \hline
         outros (\textbf{o}) & outros usuários \\
      \end{tabular}
   \end{table}
   \begin{center}
   Leitura(r), escrita(w) e execução(r).
   \end{center}
\end{frame}

\begin{frame}[fragile]
   \frametitle{chmod - Ajustando as Permissões de Arquivo}
   \begin{minted}{bash}
   $ chmod string-de-acesso arquivo
   \end{minted}
   \begin{table}
      \begin{tabular}{ c | c }
         string-de-acesso & formato padrão de permissões \\
         \hline 
         arquivo & nome dos arquivos ou diretórios\\ 
      \end{tabular}
   \end{table}
\end{frame}

\begin{frame}[fragile]
   \frametitle{who - Usuários Ativos}
   \begin{minted}{bash}
   $ who [-mH] [am i]
   \end{minted}
   \begin{table}
      \begin{tabular}{ c | c }
         am i & informações do próprio usuário \\
         \hline 
         -H & mostra um cabeçalho \\ 
         \hline
	 -a & informações detalhadas \\ 
         \hline
      \end{tabular}
   \end{table}
\end{frame}

\begin{frame}[fragile]
   \frametitle{id - Identificadores do Usuário}
   \begin{minted}{bash}
   $ id [-ngu]
   \end{minted}
   \begin{table}
      \begin{tabular}{ c | c }
         -n & informações do próprio usuário \\
         \hline 
         -g & mostra apenas o grupo \\ 
         \hline
	 -G & mostra todos os grupos \\ 
         \hline
	 -u & apenas o identificador do usuário \\ 
      \end{tabular}
   \end{table}
\end{frame}

\begin{frame}[fragile]
   \frametitle{chfn - Mudar as informações do usuário}
   \begin{minted}{bash}
   $ chfn [-frpho]
   \end{minted}
   \begin{table}
      \begin{tabular}{ c | c }
         -f & modifica o nome do usuário \\
         \hline 
         -r & modifica a localização do usuário \\ 
         \hline
	 -p & modifica o telefone de trabalho do usuário \\
         \hline
	 -h & modifica o telefone residencial do usuário \\ 
         \hline
         -o & informações adicionais (somente root) \\ 
      \end{tabular}
   \end{table}
\end{frame}

\begin{frame}[fragile]
   \frametitle{date - Exibe altera a data}
   \begin{minted}{bash}
   $ date [MMDDhhmm] [+formato]
   \end{minted}
   \begin{table}
      \begin{tabular}{ c | c }
         [MMDDhhmm] & atualiza a data \\
         \hline 
         +formato & indica como exibir a data \\ 
      \end{tabular}
   \end{table}
\end{frame}

\begin{frame}[fragile]
   \frametitle{ssh - Login Remoto Criptografado}
   \begin{minted}{bash}
   $ ssh [-p PORTA] [usuario]@IP
   \end{minted}
   \begin{table}
      \begin{tabular}{ c | c }
         -p & Tentar se conectar na PORTA no lugar da 22 (padrão) \\
         \hline 
         usuario & usuário na máquina remota \\ 
      \end{tabular}
   \end{table}
\end{frame}

\begin{frame}[fragile]
   \frametitle{scp - Copiar Arquivo Através do ssh}
   \begin{minted}{bash}
$ scp [-P PORTA] [-r] arquivo [usuario]@IP:[diretorio]
   \end{minted}
   \begin{table}
      \begin{tabular}{ c | c }
         -P & Tentar se conectar na PORTA no lugar da 22 (padrão) \\
         \hline 
         -r & cópia recursiva \\ 
         \hline
         diretorio & destino onde será copiado o arquivo no servidor remoto \\ 
      \end{tabular}
   \end{table}
\end{frame}

\begin{frame}[fragile]
   \frametitle{tar - Agrupar Arquivos}
   \begin{minted}{bash}
   $ tar [-cfprtuvx] [arquivo_tar] [arquivos]
   \end{minted}
   \begin{table}
      \begin{tabular}{ c | c }
         -c & Cria novo arquivo \\
         \hline 
         -f & Indica que o destino é o disco \\ 
         \hline
	 -p & Preserva permissões \\
         \hline 
         -r & Anexa ao final de um arquivo .tar existente \\ 
         \hline
         -t & Lista conteúdo de um arquivo .tar \\ 
         \hline
         -u & Adiciona apenas arquivos novos ou modificados  \\
         \hline 
         -v & Mostra o nome de cada arquivo processado \\ 
         \hline
         -x & Retira os arquivos do .tar \\ 
      \end{tabular}
   \end{table}
   \begin{minted}{bash}
   # Compacta pasta teste
   $ tar -czvf teste.tar.gz teste/
   # Descompacta arquivo
   $ tar -xzvf teste.tar.gz
   \end{minted}
\end{frame}

%   \begin{frame}
%      \frametitle{Exercício de Comandos Básicos - Parte I}
%      Esta atividade é para ser feita \textbf{inicialmente} na sua máquina local.
%      \begin{itemize}
%         \item Crie um diretório chamado \textit{ufc\_quixada}.
%	      \item Crie um subdiretório chamado \textit{redes\_de\_computadores}.
%	      \item Em \textit{redes\_de\_computadores}, crie um subdiretório chamado \textit{grade\_curricular}.
%         \item Acesse \url{http://rc.quixada.ufc.br/matriz-curricular/} e anote o nome cada disciplina do curso.
%         \item Para cada disciplina:
%	      \begin{itemize}
%	         \item Crie um arquivo de texto em \textit{ufc\_quixada/redes\_de\_computadores/grade\_curricular}.
%	         \item O nome do arquivo deve ser em minúsculas e os espaços devem ser trocados por \_.
%	         \item Por exemplo: Internet e Arquitetura TCP/IP corresponde ao arquivo \textit{internet\_e\_arquitetura\_tcpip}
%	         \item Dentro do arquivo, coloque o código da disciplina. 
%         \end{itemize}
%         \item \textbf{Se você for de outro curso, adapte de acordo com o nome do seu curso}.
%      \end{itemize}
%   \end{frame}
%
%   \begin{frame}
%      \frametitle{Exercício de Comandos Básicos - Parte II}
%      Este exercício é continuação da Parte I.
%      \begin{itemize}
%         \item Acesse \url{www.quixada.ufc.br} e anote o nome e o primeiro sobrenome de cada professor que você já foi aluno.
%         \item Crie um diretório \textit{ufc\_quixada/redes\_de\_computadores/professores}.
%         \item Dentro do diretório criado, crie um arquivo para cada professor. O nome do arquivo de cada professor deve seguir o padrão \textit{nome\_sobrenome}. 
%         \item Exemplo: deve existir o arquivo \textit{ufc\_quixada/rede\_de\_computadores/professores/joao\_marcelo}.
%         \item Dentro do arquivo, coloque a url da página do professor em \url{https://www.quixada.ufc.br/docente/}.
%      \end{itemize}
%      Até aqui, você terá pastas para as disciplinas em \textit{grade\_curricular} e pasta para os professores em \textit{professores}.
%   \end{frame}
%
%   \begin{frame}
%      \frametitle{Exercício de Comandos Básicos - Parte III}
%      Este exercício é continuação da Parte II.
%      \begin{itemize}
%         \item Crie o diretório \textit{ufc\_quixada/redes\_de\_computadores/conquistas}.
%	      \item Dentro de \textit{conquistas}, você vai criar um diretório para cada disciplina que já foi aprovado(a).
%         \item Dentro do diretório criado, defina dois \textit{soft links}:
%	      \begin{itemize}
%	         \item \textit{programa}, que deve apontar para o arquivo da disciplina em \textit{ufc\_quixada/redes\_de\_computadores/grade\_curricular}.
%	         \item \textit{professor}, que deve apontar para o arquivo do professor em \textit{ufc\_quixada/redes\_de\_computadores/professores}.
%	      \end{itemize}
%      \end{itemize}
%   \end{frame}
%
%   \begin{frame}
%      \frametitle{Exercício de Comandos Básicos - Parte IV}
%      Este exercício é continuação da Parte III.
%      \begin{itemize}
%         \item No seu repositório crie a pasta \textit{exercicios}.
%         \item Dentro da pasta \textit{exercicios} crie o diretório \textit{exercicio01}.
%         \item Coloque sua pasta \textit{ufc\_quixada} no repositório na pasta \textit{exercicios/exercicio01}.
%	      \item No servidor, dentro do seu diretório de usuário, clone ou atualize a cópia do repositório no diretório de usuário.
%      \end{itemize}
%      \textbf{Importante:} os \textit{links} criados devem continuar funcionando no servidor e em qualquer lugar que o repositório for clonado.
%      Tenha cuidado nos \textbf{caminhos relativos}.
%   \end{frame}
%
%   \begin{frame}
%      \frametitle{Resultado do Exercício}
%      \begin{itemize}
%         \item Pela lógica, deve existir a pasta \textit{exercicios/exercicio01/ufc\_quixada} na sua pasta no servidor.
%         \item Dentro dela, deve existir a pasta \textit{redes\_de\_computadores} com os seguintes subdiretórios:
%         \begin{itemize}
%            \item \textit{grade\_curricular} com subdiretórios para todas as disciplinas.
%	         \item \textit{professores} com subdiretórios para todos os professores.
%	         \item \textit{conquistas} com pastas para as disciplinas que foi aprovado e dentro delas ligações (\textit{links}) para os arquivos no diretório \textit{grade\_curricular} e \textit{professores}.  
%         \end{itemize}
%      \end{itemize}
%      \begin{block}{Objetivo desse trabalho todo...}
%      Garantir que o aluno tem prática no uso da linha de comando.
%      \end{block}
%   \end{frame}
%
%   \begin{frame}
%      \frametitle{Exercícios - Parte I}
% 
%      \begin{itemize}
%         \item Acesse \url{www.quixada.ufc.br} e anote o nome e o primeiro sobrenome de cada professor.
%         \item Crie um diretório chamado \textit{programacaodescripts}.
%         \item Dentro do diretório criado, crie um diretório para cada professor. O nome do diretório de cada professor deve seguir o padrão \textit{nomesobrenome}. 
%         \item Exemplo: deve existir o diretório \textit{programacaodescripts}/\textit{joaomarcelo}
%      \end{itemize}
%   \end{frame}
%
%   \begin{frame}
%      \frametitle{Exercícios - Parte II}
%
%      \begin{itemize}
%         \item Acesse o SI3 ou o SIPPA e anote o nome e o primeiro sobrenome de cada aluno da turma.
%	 \item Dentro do diretório \textit{programacaodescripts} do exercício anterior, crie um diretório para cada aluno seguindo o padrão \textit{nomesobrenome}.
%      \end{itemize}
%   \end{frame}
%
%   \begin{frame}
%      \frametitle{Exercícios - Parte III}
%
%      \begin{itemize}
%         \item Crie o subdiretório \textit{professores} dentro de \textit{programacaodescripts} e copie todos os diretórios com nomes de professores para dentro do novo diretório.
%         \item Crie o subdiretório \textit{alunos} dentro de \textit{programacaodescripts} e copie todos os diretórios com nomes de alunos para dentro do novo diretório.
%      \end{itemize}
%   \end{frame}
%
%   \begin{frame}
%      \frametitle{Exercícios - Parte IV}
%      
%      \begin{itemize}
%         \item Crie o diretório \textit{programacaodescripts/professores/enfrentados}.
%         \item Crie um link simbólico para cada professor que você já cursou alguma disciplina do diretório \textit{programacaodescripts/professores} dentro do subdiretório \textit{programacaodescripts/professores/enfrentados}.
%      \end{itemize}
%   \end{frame}

%   \begin{frame}
%      \frametitle{Exercícios - Parte V}
%      
%      \begin{itemize}
%         \item Encontre todos os arquivos maiores do que 1GB.
%         \item Encontre todos os arquivos que não foram modificados há mais de uma semana. 
%         \item Encontre todos os arquivos maiores do que 500 MB que não são acessados há mais de uma semana.
%         \item Liste o tamanho de todos os arquivos do sistema que ocupam mais do que 100 MB.
%         \item Dado um diretório com um projeto na linguagem C, procure todos os arquivos que possuem uma variável com determinado nome.
%      \end{itemize}
%   \end{frame}
 
\end{document}


