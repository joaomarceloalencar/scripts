\documentclass{beamer}

\usepackage[portuguese]{babel}
\usepackage[utf8]{inputenc}
\usepackage{minted}

\title{Variáveis e Parâmetros}
\author[João Marcelo Uchôa de Alencar]{João Marcelo Uchôa de Alencar}
\institute{Universidade Federal do Ceará - Quixadá}

\begin{document}
   \begin{frame}
      \titlepage
   \end{frame}

   \begin{frame}
      \frametitle{Variáveis}
      O nome de uma variável é iniciado por uma letra ou um sublinhado, seguido ou não por quaisquer caracteres alfanuméricos ou caracteres sublinhados.
      \begin{itemize}
         \item Para criar uma variável basta colocar o nome, seguido de um sinal de igual $=$ \textbf{colado} ao nome escolhido e, \textbf{colado} ao $=$, o valor estipulado.
	      \item Para exibirmos o valor de uma variável, devemos preceder o seu nome por um cifrão \$. Podemos e devemos utilizar \{ e \} para delimitar o nome de uma variável.
      \end{itemize}
   \end{frame}

   \begin{frame}
      \frametitle{Passando e Recebendo Parâmetros}
      Os parâmetros passados ($param_ {1}$, $param_ {2}$ ... $param_ {n}$) dão origem, dentro do programa que os recebe, às variáveis \$1, \$2, \$3 ... \$n. É importante notar que a passagem de parâmetro é posicional, isto é, o primeiro parâmetro é \$1, o segundo \$2 e assim até o \$9.
      \begin{itemize}
         \item Do 10 em diante, use \$\{10\} no lugar de \$10.
	      \item A variável \$0 é o nome do programa.
	      \item A variável \$\# é quantidade de parâmetros recebida.
	      \item A variável \$* contém o conjunto de todos os parâmetros separados por espaços em branco.
      \end{itemize}
   \end{frame}

\begin{frame}[fragile]
   \frametitle{Comando \textit{xargs}}
   Como procurar, em um diretório, todos os arquivos que possuam uma determinada cadeia de caracteres? 
   \begin{minted}{bash}
$ ls | grep -l [cadeia] 
   \end{minted}
   Isso vai funcionar? Estou procurando no nome dos arquivos ou no conteúdo dos mesmos? \\
   O correto é passar cada arquivo encontrado por \textbf{ls} em parâmetro para o \textbf{grep}. 
   \begin{minted}{bash}
$ ls | xargs grep -l [cadeia] 
   \end{minted}
   Você consegue imaginar por que é melhor do que o comando abaixo? 
   \begin{minted}{bash}
$ find [diretorio] -type f -exec grep -l [cadeia] {} \;
   \end{minted}
\end{frame}

\begin{frame}[fragile]
   \frametitle{Comando \textit{xargs}}
   Como excluir apenas os arquivos que pertencem a determinado usuário? 
   \small
   \begin{minted}{bash}
$ ls -l | grep jmhal | tr -s '' | cut -f9 -d'' | xargs rm  
   \end{minted}
   \normalsize
   Exibindo mais informações:
   \scriptsize
   \begin{minted}{bash}
$ find . -type f -name "*.txt" | xargs -i bash -c "echo removendo {}; rm {};" 
   \end{minted}
\end{frame}

%   \begin{frame}
%      \frametitle{Atividades -  Parte I}
%      Coloque os \textit{scripts} criados em \textit{atividades/atividades05}:
%      \begin{itemize}
%         \item Escreva um script chamado \textbf{addN.sh} que recebe uma lista de números e retorna a soma deles.
%	 \item Escreva um script chamado \textbf{menorMediaMaior.sh} que recebe uma lista de números e imprime na tela o menor deles, a média aritmética e o maior em outra linha. 
%      \end{itemize}
%   \end{frame}

%   \begin{frame}
%      \frametitle{Atividades - Parte II}
%      \scriptsize
%      Coloque os \textit{scripts} e arquivos criados em \textit{atividades/atividades05}. Considere um arquivo \textit{emails\_database.txt} que armazena usuários e e-mails com o seguinte formato para a linha: \\
%      \\
%      nome completo:email \\ \\
%      Escreva os seguintes \textit{scripts}:
%      \begin{itemize}
%         \item Script \textbf{addUser.sh}: recebe como parâmetro o nome completo e email e os adiciona no arquivo. Se o arquivo não existir, deve ser criado.
%	 \item Script \textbf{remUser.sh}: recebe como parâmetro o e-mail e remove a linha correspondente do arquivo.
%	 \item Script \textbf{showUser.sh}: recebe como parâmetro o e-mail e exibe o nome do usuário.
%	 \item Script \textbf{showDomains.sh}: recebe como parâmetro um domínio (gmail, yahoo, hotmail, ufc, etc) e retorna quantos usuários tem e-mail nesse domínio (ou semelhante) e exibe esses usuários.
%	 \item Script \textbf{showFile.sh}: exibe o conteúdo do arquivo.
%      \end{itemize}
%   \end{frame}

%   \begin{frame}
%      \frametitle{Atividades - Parte III}
%      Coloque os \textit{scripts}, arquivos e pastas criados em \textit{atividades/atividades07}. Considere todos os arquivos do diretório /home/compartilhado/maillog: 
%      \begin{itemize}
%         \item Os arquivos não estão organizados por dia. Por exemplo mail.log.1 tem informação do dia 05\slash09 até o dia 11\slash09. Crie novos arquivos no formato mail\_DD\_MM.log de forma que cada arquivo contenha apenas as entradas do dia DD do mês MM. É necessário que os arquivos de entrada sejam fornecidos através de parâmetro.
%	 \item Use o \textbf{xargs} para executar um comando que transfira os arquivos do 09 para o diretório 09, os arquivos do mês 08 para o diretório, etc. Por enquanto são apenas dois meses, mas faça seu comando para suportar quantidade maior.
%      \end{itemize}
%   \end{frame}

%   \begin{frame}
%      \frametitle{Exercícios - Parte IV}
%      Considerando um arquivo emails\_database.txt criado de acordo com o Exercício II. \\
%      Crie um script chamado \textbf{sendDomain.sh} que receba três parâmetros: o domínio, um nome de um arquivo contendo uma frase de assunto e o nome de um arquivo que contém uma mensagem. Envie uma mensagem de e-mail com o assunto e o texto informados para todos os e-mails do domínio presentes no arquivo. Para simplificar o exercício, coloque dois e-mails de um mesmo domínio no arquivo e execute o script.
%   \end{frame}

\end{document}
