\documentclass{beamer}

\usepackage[portuguese]{babel}
\usepackage[utf8]{inputenc}
\usepackage{minted}

\title{Redirecionamento de Saída}
\author[João Marcelo Uchôa de Alencar]{João Marcelo Uchôa de Alencar}
\institute{Universidade Federal do Ceará - Quixadá}

\begin{document}
   \begin{frame}
      \titlepage
   \end{frame}

\begin{frame}[fragile]
   \frametitle{Incorporando Resultados de Comandos}
   Utilizando crases ou \$() podemos incorporar o resultado de um comando na execução de outro. \\
   \begin{minted}{bash}
   $ echo "O nome da máquina é $(hostname)" 
   $ echo "O nome da máquina é `hostname`" 
   \end{minted}
   Um novo processo \textit{shell} é criado, executa o comando e retorna o resultado.
\end{frame}
   
   \begin{frame}
      \frametitle{Redirecionamento de Saída}
      \begin{table}
         \begin{tabular}{ c | l }
         $>$ & Redireciona a saída de um comando para um arquivo. \\
         \hline 
         $>>$ & Anexa saída a um arquivo. \\
         \hline 
         2$>$ & Redireciona erros gerados para um arquivo. \\
         \hline 
         \end{tabular}
      \end{table}
   \end{frame}

   \begin{frame}
      \frametitle{Redirecionamento de Entrada}
      \begin{table}
         \begin{tabular}{ c | l }
         $<$ & Fornece entrada a um comando a partir de um arquivo. \\
         \hline 
         $<<$ & Permite inserção de múltiplas linhas. \\
         \hline 
         \end{tabular}
      \end{table}
   \end{frame}

  \begin{frame}
      \frametitle{Redirecionamentos Especiais}
      \begin{table}
         \begin{tabular}{ c | l }
         $|$ & Permite redirecionar a saída de um comando para outro. \\
         \hline 
         tee & Salva a saída em um arquivo mas também exibe na tela. \\
         \hline 
         \end{tabular}
      \end{table}
   \end{frame}

   \begin{frame}[fragile]
      \frametitle{Qual o efeito dos seguintes comandos?}
      \begin{minted}{bash}
ls | wc -l 
ls > ~/saida.log 2> ~/error.log 
ls Nuncavi > ~/saida.log 2> ~/error.log
echo Nome do Sistema: uname  
echo Nome do Sistema: `uname`
tail -n 50 /home/compartilhando/auth.log | head -n 25  
       \end{minted}
\end{frame}

%\begin{frame}[fragile]
%   \frametitle{Exercícios 03 - Parte I}
%   Coloque os arquivos no diretório \textit{exercicios/exercicios03}. Escreva um \textit{script} chamado \textit{saudacao.sh} que imprima
%   a seguinte saudação ao ser executado:
%   \begin{minted}{bash}
%   $ ./saudacao.sh
%   Olá jmhal,
%   Hoje é dia 30, do mês 08 do ano de 2016.
%   \end{minted}
%   Instruções:
%   \footnotesize
%   \begin{itemize}
%      \item No lugar de \textit{jmhal}, deve ser exibido o usuário que está executando o \textit{script}.
%      \item A data exibida deve ser a data atual da execução, sendo que o \textit{script} deve funcionar sem alterações seja qual for o dia que for executado.
%      \item Você deve usar os comandos \textbf{echo}, \textbf{who} e \textbf{date}. 
%      \item Veja mais informações sobre o \textit{date} em \url{https://www.vivaolinux.com.br/artigo/Formatando-exibicao-de-datas-no-Linux}.
%   \end{itemize}
%   \normalsize
%   Por último, toda vez que o \textit{script} for executado, além de exibir a saudação na tela, deve anexar
%   a saudação ao arquivo \textit{saudacao.log} no mesmo diretório de execução.
%\end{frame}

%   \begin{frame}
%      \frametitle{Exercícios - Parte I}
%      Usando os comandos \textbf{echo} e \textbf{date}, imprima a seguinte frase: \\
%      \textit{Hoje é dia 30, do mês 08 do ano de 2016.} \\
%      sem digitar números na \textit{string} passada ao \textbf{echo}.
%   \end{frame}

%  \begin{frame}
%     \frametitle{Atividade}
%     Crie o diretório \textit{atividades/atividade04}. Copie o arquivo \textit{/home/compartilhado/auth.log} para o diretório criado. Crie um arquivo chamado \textit{auth\_analise.txt} com os seguintes comandos, um por linha:
%     \begin{enumerate}
%        \item Um comando que analise o arquivo \textit{auth.log} e informe quantas vezes tentaram fazer \textit{login} como um usuário inválido.
%	       \item Um comando que salve em um arquivo chamado \textit{malditos.txt} todas as tentativas de \textit{login} como usuários inválidos.
%	       \item Vasculhe o arquivo até encontrar um usuário que também seja aluno. Depois escreva o comando que conte quantas vezes ele ou ela fez o \textit{login} com sucesso. Pode usar seu usuário. 
%     \end{enumerate}
%     Abra o arquivo \textit{auth.log} em um editor de texto e estude seu formato antes de tentar definir os comandos.
%  \end{frame}

%   \begin{frame}
%      \frametitle{Enviado E-mail no Bash}
%      Para enviar: \\
%      \$ mail -s "Assunto" [endereço] [mensagem] \\
%      Para ler todos: \\
%      \$ mail $--$read
%   \end{frame}

%   \begin{frame}
%      \frametitle{Exercícios - Parte III}
%      Utilizando o redirecionamento de entrada ($<<$), faça o seguinte:

%      \begin{itemize}
%        \item Envie um e-mail do servidor para seu e-mail. Lembre que pode cair no SPAM.
%	      \item Responda o e-mail e veja a resposta no terminal.
%	      \item Cada colega seu no servidor também tem um e-mail @scripts.joao.marcelo.nom.br. Envie um e-mail para um colega.
%      \end{itemize}
%   \end{frame}

\end{document}

