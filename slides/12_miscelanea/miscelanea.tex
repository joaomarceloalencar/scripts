\documentclass{beamer}

\usepackage[portuguese]{babel}
\usepackage[utf8]{inputenc}
\usepackage{minted}

\title{Miscelânea}
\author[João Marcelo Uchôa de Alencar]{João Marcelo Uchôa de Alencar}
\institute{Universidade Federal do Ceará - Quixadá}

\begin{document}
   \begin{frame}
      \titlepage
   \end{frame}

   \begin{frame}[fragile]
      \frametitle{O comando \textit{eval}}
      \begin{minted}{bash}
$ cano='|'
$ ls $cano wc -l
ls: cannot access '|': No such file or directory
ls: cannot access 'wc': No such file or directory
$ eval ls $cano wc -l
5
      \end{minted}

      \begin{itemize}
         \item Permite avaliar o conteúdo de uma variável como se fosse um comando.
         \item Você pode ir construindo um comando em uma variável para depois executá-lo.
      \end{itemize}
   
\end{frame}

   \begin{frame}[fragile]
      \frametitle{O comando \textit{wait}}
      \begin{minted}{bash}
$ sleep 60 &
[1] 3223
$ wait 3223
      \end{minted}
      \begin{itemize}
         \item Bloqueia até o processo termine.
         \item Você pode criar vários processos e usar \textit{wait} para ter certeza que todos terminaram.
      \end{itemize}
\end{frame}

   \begin{frame}
      \frametitle{O comando \textit{trap}}
      \begin{small}
 \$ trap "[comando1];[comando2];[comando3]; ... ;[comandoN]" sinais \\
      \end{small}   
      \begin{table}
         \begin{tabular}{c|l}
         0 EXIT     & Saída normal \\
         \hline 
         1 SIGHUP   & kill -HUP \\
         \hline 
         2 SIGINT   & CTRL + C \\
         \hline 
         3 SIGQUIT  & CTRL + \textbackslash \\
         \hline 
         15 SIGTERM & kill PID \\
         \hline 
         \end{tabular}
      \end{table}
      \begin{itemize}
         \item Permite executar comandos de acordo com os sinais recebidos.
         \item Pode ser usado para limpar arquivos temporários.
      \end{itemize}
   \end{frame}


   \begin{frame}[fragile]
      \frametitle{Funções}
      \begin{minted}{bash}
funcao ()
{
   comando1
   comando2
   comando3
   ...
   comandoN
   return $var # O valor de var deve ser número
}
   \end{minted}
      Invocação:
      \begin{minted}{bash}
funcao [parametro1] [parametro2] ... [parametroN] 
      \end{minted}
      Cada parâmetro dentro da função é tratado por \$1 \$2 \$3 ...
\end{frame}

%%   \begin{frame}
%%      \frametitle{Funções}
%%      \begin{center}
%%         \includegraphics[scale=0.5]{estruturascripts.png}
%%      \end{center}
%%   \end{frame}

   \begin{frame}[fragile]
      \frametitle{Funções}
      \begin{minted}{bash}
function command_not_found_handle
{
   echo "Erro na linha ${BASH_LINENO[0]}"
   exit      
}
      \end{minted}
      \begin{itemize}
         \item Permite tratar erros de execução de comandos.
         \item O comando \textit{exit} encerra a execução do \textit{script}.
      \end{itemize}
\end{frame}

%   \begin{frame}
%      \frametitle{Exercícios em Sala}
%      O objetivo é criar um \textit{script} chamado \textit{sistema.sh} para permitir monitorar o desempenho de um servidor Linux. 
%     \begin{itemize} 
%        \item Ele deve exibir um menu com opções para o usuário. 
%        \item Ao digitar uma das opções, a tela deve ser limpa, um comando executado, o resultado exibido. 
%        \item Após o usuário apertar \textit{enter} retornar para o menu inicial.
%     \end{itemize}
%     As opções devem ser, de acordo com \url{http://techblog.netflix.com/2015/11/linux-performance-analysis-in-60s.html}:
%     \begin{enumerate}
%        \item Tempo ligado (uptime)
%	      \item Últimas Mensagens do Kernel (dmesg $|$ tail -n 10)
%	      \item Memória Virtual (vmstat 1 10)
%	      \item Uso da CPU por núcleo (mpstat -P ALL 1 5)
%	      \item Uso da CPU por processos (pidstat 1 5)
%	      \item Uso da Memória Física (free -m)
%	      \item Sair
%     \end{enumerate}
%     Use funções para criar uma função chamada \textit{menu} para exibir as opções.
%   \end{frame}

   \begin{frame}[fragile]
      \frametitle{O comando \textit{mkfifo}}
      \begin{minted}{bash}
# No terminal 1
$ mkfifo cano
$ ls > cano
# No terminal 2
$ cat < cano
cano  sistema.sh  trap.sh
      \end{minted}
      \begin{itemize}
         \item Cria um \textit{pipe} nomeado.
         \item Permite comunicação e sincronização entre processos.
      \end{itemize}
\end{frame}

   \begin{frame}[fragile]
      \frametitle{O Comando \textit{getopts}}
      \begin{minted}{bash}
      $ getopts [cadeia de opções] variável 
      \end{minted}
      \begin{itemize}
         \item Se o \textit{script} aceita as opções -a -b -c, então a cadeia de opções deve ser $abc$. 
         \item Se uma das opções tem argumento, : deve ser colocado depois da letra, a:bc 
         \item A variável irá armazenar qual opção está sendo tratada, enquanto OPTARG contém o valor passado.
      \end{itemize}	 
\end{frame}
   
   \begin{frame}[fragile]
      \frametitle{O Comando \textit{getopts}}
     \begin{minted}{bash}
$ cat getopts_exemplo.sh 
#!/bin/bash
while getopts "a:bc" OPTVAR
do
   echo $OPTVAR $OPTARG
done
$ ./getopts_exemplo.sh -a 1 -b -c
a 1
b
c
$ ./getopts_exemplo.sh -b -c -a 1  
b
c
a 1
      \end{minted}
\end{frame}

   \begin{frame}
      \frametitle{Execução Passo a Passo}
      Muitas vezes não é fácil encontrar o erro em \textit{Shell Scripts}.
      \begin{itemize}
         \item Uma opção é ativar a execução passo a passo.
	      \begin{itemize}
	         \item Coloque \textit{set -x} no início do trecho que deseja analisar.
	         \item No fim do trecho desabilite com \textit{set +x}. 
	      \end{itemize}
	      \item Se você quiser listar as linhas com mais detalhes.
	      \begin{itemize}
	         \item Ativar com \textit{set -xv}.
	         \item Desabilitar com \textit{set +xv}.
	      \end{itemize}
      \end{itemize}
   \end{frame}

   \begin{frame}[fragile]
      \frametitle{Exemplo de Execução Passo a Passso}
      \begin{minted}{bash}
#!/bin/bash
echo "Iniciando execução passo a passo."
set -x
echo "Flag ativado."
read -p "Informe um valor:" entrada
entrada=`expr $entrada + 1`
echo "Valor atualizado: $entrada"
set +x
echo "Flag desativado."
      \end{minted}
\end{frame}

   \begin{frame}[fragile]
      \frametitle{Exemplo de Execução Passo a Passso}
      \begin{minted}{bash}
#!/bin/bash
echo "Iniciando execução passo a passo."
set -xv
echo "Flag ativado."
read -p "Informe um valor:" entrada

soma=0
for i in `seq $entrada`
do
   echo "Atualizando entrada..."
   soma=`expr $soma + $i`
done

echo "Valor atualizado: $soma"
set +xv
echo "Flag desativado."
      \end{minted}
\end{frame}


%   \begin{frame}
%      \frametitle{Exercícios em Sala}
%      Vamos criar um \textit{script} chamado \textit{hosts} que nos ajude a relacionar nomes de máquinas à IPs.
%      \begin{itemize}
%        \item O \textit{script} deve guardar em um arquivo chamado \textit{hosts.db} um par (nomedamaquina,IP) para cada entrada.
%	      \item Você deve criar as seguintes funções para manipular o arquivo que são invocadas com os parâmetros indicados:
%         \begin{itemize}
%            \item \textbf{adicionar} (parâmetros -a \textit{hostname} -i \textit{IP})
%	         \item \textbf{remover} (parâmetro -d \textit{hostname})
%	         \item \textbf{procurar} (parâmetro \textit{hostname})
%	         \item \textbf{listar} (parâmetro -l)
%         \end{itemize}
%      \end{itemize}
%   \end{frame}
%
%   \begin{frame}[fragile]
%      \frametitle{Exercícios em Sala}
%      \begin{minted}{bash}
%      $ ./hosts -a routerlab -i 192.168.0.1
%      $ ./hosts -a lab01 -i 192.168.0.100
%      $ ./hosts -a lab02 -i 192.168.0.101
%      $ ./hosts -l
%      routerlab 192.168.0.1
%      lab01     192.168.0.100
%      lab02     192.168.0.101
%      lab03     192.168.0.102
%      $ ./hosts -d routerlab
%      $ ./hosts -d lab01
%      $ ./hosts -l 
%      lab02     192.168.0.101
%      lab03     192.168.0.102
%      $ ./hosts lab02
%      192.168.0.101
%      $ ./hosts -r 192.168.0.101
%      lab02
%      \end{minted}
%\end{frame}
%
%   \begin{frame}
%      \frametitle{Atividade Especial}
%      Agora que já conhecemos o comando \textit{getopts} e funções:
%      \begin{itemize}
%        \item Refaça a segunda questão da prova melhorando sua solução usando os conhecimentos desta aula.
%	      \item Obrigatoriamente pelo menos o \textit{getopts} deve ser usado.
%	      \item Vale 1,0 ponto na nota da prova.
%	      \item O nome do \textit{script} deve ser \textit{questao02\_extra.sh}.
%	      \item Deixe na mesma pasta da prova.
%      \end{itemize}
%   \end{frame}

%  \begin{frame}
%     \frametitle{Atividades}
%     Refaça o exercício de sala anterior usando o nome \textit{hosts.sh} na pasta \textit{atividades/atividade13}, porém caso o usuário não forneça nenhuma opção por parâmetros, o \textit{script} deve exibir um menu com as opções disponíveis (adicionar, remover, procurar e listar). Se a opção envolver leitura de valores, deve-se requisitar a entrada adequada do usuário.
%  \end{frame}

\end{document}
