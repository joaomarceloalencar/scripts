\documentclass{beamer}

\usepackage[portuguese]{babel}
\usepackage[utf8]{inputenc}
\usepackage{minted}

\title{Gerência de Processos}
\author[João Marcelo Uchôa de Alencar]{João Marcelo Uchôa de Alencar}
\institute{Universidade Federal do Ceará - Quixadá}

\begin{document}
   \begin{frame}
      \titlepage
   \end{frame}

   \begin{frame}[fragile]
      \frametitle{O comando \textit{ps}}
      \begin{itemize}
         \item Exibe informações detalhadas dos processos em execução no sistema:
      \end{itemize}
      \begin{minted}{bash}
      $ ps aux 
      \end{minted}
      \begin{itemize}
         \item Se quiser saber se um script com nome \textit{meuscript.sh} está em execução?
      \end{itemize}
      \begin{minted}{bash}
      $ ps aux | grep meuscript
      \end{minted}
      \begin{itemize}
         \item Se quero ver todos os processos do meu usuário?
      \end{itemize}
      \begin{minted}{bash}
      $ ps aux | grep usuario
      \end{minted}
      \begin{itemize}
         \item Detalhe: o \textit{grep} é um comando que filtra a saída de um comando.
         \item Para matar um processo, existe o comando \textit{kill}.
      \end{itemize}
      \begin{minted}{bash}
      $ kill PID
      \end{minted}
      \begin{itemize}
         \item O PID é a segunda coluna da saída do \textit{ps aux}.
      \end{itemize}
\end{frame}

   \begin{frame}[fragile]
      \frametitle{O uso do \&}
      Caso você deseje executar um comando e ter de volta o terminal?
      \begin{minted}{bash}
$ ./meuscript.sh [parametros] [redirecionamentos] &
      \end{minted}
      \begin{itemize}
         \item Você pode, a partir do seu \textit{script}, iniciar processos da mesma forma.
         \item A variável \$! armazena o PID do último processo criado. 
         \item Se você terminar a sessão, ou seu \textit{script} finalizar, o que acontece com os processos criados depende da distribuição. Eles podem ser finalizados ou continuar. 
         \item Para garantir que eles continuam a executar, use o \textit{nohup}. 
      \end{itemize}   
      A saída do \textit{script} será direcionada para \textit{nohup.out} e você pode ter certeza que ele continuará executando até ser finalizado pelo \textit{kill}.
\end{frame}

\begin{frame}[fragile]
   \frametitle{O uso do \textit{nohup}}
   \begin{minted}{bash}
$ nohup ./meuscript.sh &
   \end{minted} 
   \begin{itemize}
      \item  A saída do \textit{script} será direcionada para \textit{nohup.out}.
      \item Você pode ter certeza que ele continuará executando até ser finalizado pelo \textit{kill}.
   \end{itemize}
\end{frame}

   \begin{frame}[fragile]
      \frametitle{O comando \textit{screen}}
      \begin{itemize}
         \item O \& e o \textit{nohup} são bons para criar \textit{daemons}.
         \item Mas se você quiser manter uma sessão ativa enquanto entra e sai na máquina, o \textit{screen} é uma boa opção:
      \end{itemize}
      \begin{minted}{bash}
$ screen
      \end{minted}        
      \begin{itemize}
         \item Você agora pode deixar qualquer comando executando. 
         \item Para \textit{desanexar} a tela, digite crtl $+$ a $+$ d.
         \item Para \textit{desanexar} a tela, digite crtl $+$ a $+$ d.
         \item Você pode fazer \textit{logout} e um novo \textit{login}. Para recuperar a sessão, digite: 
      \end{itemize}      
      \begin{minted}{bash}
$ screen -r
      \end{minted} 
\end{frame}

   \begin{frame}
      \frametitle{Mais opções do \textit{screen}}
      \begin{itemize}
         \item Criar um terminal em uma tela: ctrl + a + c 
         \item Trocar entre as telas: ctrl + a + tab 
         \item Ir para a próxima tela: ctrl + a + espaço 
         \item Ir para a tela anterior: ctrl + a + backspace 
         \item Ir para a n-ésima tela: ctrl + a + [n] 
         \item Listar as telas abertas: ctrl + a + \textquotedblleft 
         \item Dividir a tela na vertical: ctrl + a + $|$ 
         \item Dividir a tela na horizontal: ctrl + a + S 
         \item Encerrar a divisão de telas: ctrl + a + Q 
      \end{itemize}
   \end{frame}

   \begin{frame}[fragile]
      \frametitle{O comando \textit{tmux}}
      \begin{itemize}
         \item É similar ao \textit{screen} porém mais recente.
         \item Inicializando uma sessão:
      \end{itemize}
   
      \begin{minted}{bash}
      $ tmux	      
      \end{minted}

      \begin{itemize}
         \item Podemos também nomear as sessões:
      \end{itemize}

      \begin{minted}{bash}
      $ tmux new -s programando	      
      \end{minted}

      \begin{itemize}
         \item Listando sessões ativas:
      \end{itemize}

      \begin{minted}{bash}
      $ tmux ls
      \end{minted}

      \begin{itemize}
         \item  Retornar a uma sessão específica:
      \end{itemize}
     
      \begin{minted}{bash}
      $ tmux attach -t 0 
      \end{minted}
\end{frame}

   \begin{frame}
      \frametitle{Mais opções do \textit{tmux}}
      \begin{itemize}
         \item Criar um novo painel: CTRL + b + c
         \item Dividir um painel: CTRL + b + \%
         \item Alternar entre painéis: CTRL + b + \textit{numero}
         \item Fechar um painel: CTRL + b + d
         \item Voltar ao painel anterior: CTRL + b + p
         \item Avançar ao próximo painel: CTRL + b + n
         \item Desanexar da sessão: CTRL + d
      \end{itemize}
   \end{frame}


% \begin{frame}
%    \frametitle{Exercício em Sala}
%    O campo VZS da saída do \textbf{ps aux} equivale a quantidade de memória virtual em \textit{kilobytes} que o processo utiliza. \\
%    Vamos fazer um \textit{script} chamado \textit{consumoDeMemoria.sh} que receba como parâmetro o nome do usuário e imprima quanto o mesmo está consumindo da memória virtual. \\
% \end{frame}

%   \begin{frame}
%     \frametitle{Exercício em Sala}
%      Vamos fazer um \textit{script} chamado \textit{IPsAtivos.sh} que receba como parâmetros os três primeiros octetos de uma sub-rede \slash 24 e verifique quais IPs da rede estão ativos. \\
%      Exemplo: \\
%      \$ ./IPsAtivos.sh 192.168.0. \\
%      192.168.0.1   on \\
%      192.168.0.54  on \\
%      192.168.0.101 on \\
%      Esse \textit{script} pode demorar muito. Use algumas das técnicas apresentadas hoje para guardar o resultado em algum lugar e depois verificá-lo. \\
%   \end{frame}

%   \begin{frame}
%      \frametitle{Atividade - Parâmetros}
%      Crie o diretório \textit{atividades/atividade07}. Escreva um \textit{script} chamado \textit{alertaDiretorio.sh} que recebe como \textbf{parâmetros} um valor inteiro que serve como intervalo de tempo em segundos e um nome que indica o caminho de um diretório. \\
%   \end{frame}
%
%   \begin{frame}
%      \frametitle{Atividade - Ações}
%      A cada intervalo, a quantidade de arquivos em um diretório será analisada. Caso a quantidade de arquivos se altere entre duas verificações, o \textit{script} deve atualizar um arquivo chamado \textit{dirSensors.log} com as seguintes informações:
%      \begin{itemize}
%         \item A data que a alteração foi percebida.
%	       \item Quantos arquivos haviam.
%         \item Quantos existem agora.
%         \item Quais foram alterados, adicionados ou removidos.
%      \end{itemize}
%      Na mesma pasta da atividade, crie um diretório chamado \textit{diretorioMonitorado} para testar. \\
%      Deixe seu \textit{script} executando em uma sessão \textit{screen} ou \textit{tmux}, monitorando o diretório a cada 5 segundos. \\
%   \end{frame}
%
%   \begin{frame}[fragile]
%      \frametitle{Atividade - Formato do Arquivo}	     
%      \footnotesize
%      \begin{minted}{bash}
%$ ./alertaDiretorio.sh 5 diretorioMonitorado
%[25-09-2018 12:59:51] Alteração! 3->2. Removidos: notas.txt
%[25-09-2018 13:04:51] Alteração! 2->4. Adicionados: a.txt, b.txt
%[25-09-2018 13:09:51] Alteração! 4->3. Removidos: a.txt
%[25-09-2018 13:14:51] Alteração! 3->2. Removidos: b.txt
%      \end{minted}
%\end{frame}
\end{document}
