\documentclass{beamer}

\usepackage[portuguese]{babel}
\usepackage[utf8]{inputenc}
\usepackage{minted}

\title{Condicionais}
\author[João Marcelo Uchôa de Alencar]{João Marcelo Uchôa de Alencar}
\institute{Universidade Federal do Ceará - Quixadá}

\begin{document}
   \begin{frame}
      \titlepage
   \end{frame}

   \begin{frame}
      \frametitle{Estruturas de Controle}
      Enquanto outras linguagens testam \textbf{condições}, o \textit{Shell} testa o código de retorno do comando que o segue. \\
      \begin{itemize}
         \item O código de retorno está associado à variável \$?
	      \item Sucesso (verdadeiro) equivale à 0.
	      \item Erro (falso) equivale à 1.
      \end{itemize}
      No caso de um $|$, é retornado uma vetor PIPESTATUS, cada posição equivale ao código de retorno do comando na posição equivalente do \textit{pipe}.
      \begin{itemize}
         \item \$\{PIPESTATUS[0]\} é o código de retorno do primeiro comando.
	      \item \$\{PIPESTATUS[*]\} exibe todo o vetor.
      \end{itemize}
   \end{frame}

   \begin{frame}[fragile]
      \frametitle{A estrutura if\slash then\slash else}
      \begin{minted}{bash}
      if <comando>
      then
         <comando1>
	      <comando2>
	      <...>
      else
         <comando3>
	      <comando4>
	      <...>
      fi
      \end{minted}
\end{frame}

   \begin{frame}
      \frametitle{Exemplo}
      Como seria um \textit{script} chamado \textbf{numeroOuNao.sh} que recebe um parâmetro e afirma se o mesmo é um número ou não.
   \end{frame}

   \begin{frame}[fragile]
      \frametitle{O comando \textit{test}}
      Como fazer para analisar condições booleanas no estilo de comparações $a > b , a == b$, etc?
      \begin{minted}{bash}
      $ test <expressao>
      \end{minted}
      Sendo \textbf{expressao} a condição a ser testada. \\
      Ao usar o \textbf{test} é bom colocar as variáveis entre aspas. Há diferença entre comparar a \textit{string} vazia e uma variável não existente.
\end{frame}

   \begin{frame}
      \frametitle{Exemplo}
      Como seria um \textit{script} chamado \textit{param.sh} que verifica se o usuário passou parâmetros ou não. Caso não tenha passado parâmetros, deve imprimir apenas o nome do \textit{script}. Caso tenha passado parâmetros, deve imprimir todos.
   \end{frame}

   \begin{frame}
      \frametitle{Testando Arquivos}
      \begin{table}
         \begin{tabular}{l|l}
	 \hline
         -r arquivo & Permissão de Leitura \\
         \hline 
         -w arquivo & Permissão de Escrita \\
         \hline 
         -x arquivo & Permissão de Gravação \\
         \hline 
         -f arquivo & Arquivo Regular \\
         \hline 
         -d arquivo & Diretório \\
         \hline 
         -s arquivo & Tamanho maior que zero \\
         \hline 
         \end{tabular}
      \end{table}
   \end{frame}
 
   \begin{frame}
      \frametitle{Testando Cadeia de Caracteres}
      \begin{table}
         \begin{tabular}{l|l}
	 \hline
         -z cad1 & Se o tamanho é zero \\
         \hline 
         -n cad1 & Se o tamanho é diferente de zero \\
         \hline 
         cad1 $=$ cad2 & Se são idênticas \\
         \hline 
	 cad1 & Se é não nula \\
	 \hline
         \end{tabular}
      \end{table}
   \end{frame}

   \begin{frame}
      \frametitle{Testando Números}
      \begin{table}
         \begin{tabular}{l|l}
	 \hline
         int1 -eq int2 & Igualdade \\
         \hline 
         int1 -ne int2 & Desigualdade \\
         \hline 
         int1 -gt int2 & Maior \\
         \hline 
         int1 -ge int2 & Maior ou igual \\
         \hline 
         int1 -lt int2 & Menor que \\
         \hline 
         int1 -le int2 & Menor ou igual \\
         \hline 
         \end{tabular}
      \end{table}
   \end{frame}

   \begin{frame}[fragile]
      \frametitle{Simplificando a escrita do \textit{test}}
      No lugar de: 
      \begin{minted}{bash}
      $ test <expressao>
      \end{minted}
      Podemos escrever:
      \begin{minted}{bash}
      $ [ <expressao> ]
      \end{minted}
      Importante notar que há um espaço entre os colchetes e a expressão. \\
      Se houver um sinal de ! antes da expressão, ela está negada. Mas lembre, tudo separado por \textbf{espaços}!!! \\
      O \textbf{e} lógico pode ser criado usando \textbf{-a} para separar as expressões. \\
      O \textbf{ou} lógico pode ser criado usando \textbf{-o} para separar as expressões. \\
      Você pode usar parênteses, mas eles devem ser precedidos da barra invertida para o shell não interpretá-los: \\
      \begin{minted}{bash}
     \( $a = $b -o $c = $d \) -a \( $a -gt $d \)
     \end{minted}
\end{frame}

   \begin{frame}[fragile]
      \frametitle{Usando o \&\& e o $||$}
      Executar o segundo comando apenas se o primeiro for sucesso: 
      \begin{minted}{bash}
      comando1 && comando2
      \end{minted}
      Executar o primeiro comando, e caso este falhe, executar o segundo: 
      \begin{minted}{bash}
      comando1 || comando2
      \end{minted}
\end{frame}

   \begin{frame}
      \frametitle{Exemplo}
      Vamos definir dois comandos:
      \begin{itemize}
         \item Caso um diretório chamado \textbf{novoDir} não exista, deve ser criado.
	 \item Mover todos os arquivos \textit{.txt} para um diretório chamados \textbf{textos} e em seguida compacte o diretório.
      \end{itemize}
   \end{frame}

   \begin{frame}[fragile]
      \frametitle{Testando Padrões}
      Você pode testar se uma cadeia casa com um padrão usando:
      \begin{minted}{bash}
      [[ expressão ]]
      \end{minted}
      Sendo que expressão obecede: \\ 
      \begin{table}
         \begin{tabular}{l|l}
	 \hline
         cadeia == padrão & Casamento \\
         \hline 
         cadeia != padrão & Incompatibilidade \\
         \hline 
         cadeia1 $<$ cadeia2 & Ordem alfabética \\
         \hline 
         cadeia1 $>$ cadeia2 & Ordem alfabética \\
         \hline 
         expr1 \&\& expr2 & Lógica \\
         \hline 
         expr1 $||$ expr2 & Lógica \\
         \hline 
         \end{tabular}
      \end{table}
\end{frame}
   
   \begin{frame}[fragile]
      \frametitle{Expressões Regulares}
      Além de padrões simples, podemos casar expressões regulares: \\
      \begin{minted}{bash}
      [[ $cadeia =~ expressaoregular ]]
      \end{minted}
      Se a expressão regular usar parêntenses, o vetor BASH\_REMATCH armazena as subcadeias que casaram com cada subexpressão. \\
      \begin{itemize}
         \item \$\{BASH\_REMATCH[0]\} é cadeira inteira
	 \item \$\{BASH\_REMATCH[1]\} é a subcadeia que casou com a subexpressão no primeiro parênteses.
      \end{itemize}
      No caso das expressões, os parênteses não precisam ser escritos precedidos por barras.
      \begin{minted}{bash}
      $ num=100
      $ [[ $num =~ ([0-9])[0-9][0-9] ]]
      \end{minted}
\end{frame}

   \begin{frame}
      \frametitle{Exemplo}
      Escrever um comando que preencha o vetor com os octetos de um endereço IP. \\
      Por exemplo, para o IP 200.19.190.1: \\
      \$\{BASH\_REMATCH[1]\} = 200 \\
      \$\{BASH\_REMATCH[2]\} = 19 \\
      \$\{BASH\_REMATCH[3]\} = 190 \\ 
      \$\{BASH\_REMATCH[4]\} = 1 \\
   \end{frame}
   
   \begin{frame}[fragile]
      \frametitle{O comando \textit{case}}
      \scriptsize
      \begin{minted}{bash}
      case <valor> in
         padr1)
            <comando1>
	    <comando2>
	    <...>
	    ;;
         padr2)
            <comando1>
	    <comando2>
	    <...>
	    ;;
	 ...
	 padrN)
            <comando1>
	    <comando2>
	    <...>
	    ;;
         *)
	    <comando1>
	    <comando2>
	    <...>
      esac 
      \end{minted}
      Os padrões só usam um conjunto básico de meta-caracteres, não são expressões regulares completas.
\end{frame}

%   \begin{frame}[fragile]
%      \frametitle{Atividade - Parte I}
%      Na pasta \textit{atividades/atividade05}, faça um \textit{script} chamado \textit{isfile.sh} que receba um parâmetro e verifique se é o nome de um arquivo ou diretório e informe se você tem permissão de escrita e leitura. \\
%      Por exemplo: 
%      \begin{minted}{bash}
%      $ ./testFile.sh /etc/hosts 
%      É um arquivo. 
%      Tem permissão de leitura. 
%      Não tem permissão de escrita. 
%      \end{minted}
%\end{frame}
%
%   \begin{frame}[fragile]
%      \frametitle{Atividade - Parte II}
%      Na pasta \textit{atividades/atividade05}, faça um \textit{script} chamado \textit{maiorDe3.sh} que receba três números como parâmetros e retorne o maior. Não pode utilizar o comando \textbf{sort}.
%      Por exemplo: 
%      \begin{minted}{bash}
%      $ ./maiorDe3.sh 4 6 5 
%      6
%      \end{minted}
%\end{frame}
%
%   \begin{frame}[fragile]
%      \frametitle{Atividade - Parte III}
%      Na pasta \textit{atividades/atividade05}, faça um \textit{script} chamado \textit{maiorDe3Verificado.sh} que receba três números como parâmetros e retorne o maior. Não pode utilizar o comando \textbf{sort}. Esse \textit{script} tem que reclamar caso um dos parâmetros não seja número.
%      Por exemplo: 
%      \begin{minted}{bash}
%      $ ./maiorDe3Verificado.sh 4 6 5
%      6
%      $ ./maiorDe3Verificado.sh casa 10 11
%      Opa!!! casa não é número.
%      \end{minted}
%\end{frame}
%
%   \begin{frame}[fragile]
%      \frametitle{Atividade - Parte IV}
%      Na pasta \textit{atividades/atividade05}, faça um \textit{script} chamado \textit{infoDir.sh} que receba como um parâmetro um caminho. Se for um diretório, deve informar o tamanho do mesmo e quantos arquivos ou subdiretórios possui. Caso contrário, deve exibir uma mensagem reclamando.
%      \begin{minted}{bash}
% $ ./inforDir.sh /etc
% O diretório /etc ocupa 1035 kilobytes e tem 45 itens.
% $ ./inforDir.sh /etc/passwd
% /etc/passwd não é um diretório!!!
%      \end{minted}
%\end{frame}
%
%   \begin{frame}
%      \frametitle{Atividade - Parte V}
%      De volta ao diretório \textit{atividades/atividade05}, escreva um \textit{script} chamado \textbf{servico.sh}. Esse script deve receber um parâmetro.
%      \begin{itemize}
%         \item Se o parâmetro for \textit{start}, deve imprimir \textquotedblleft Iniciando Serviço\textquotedblright.
%         \item Se o parâmetro for \textit{stop}, deve imprimir \textquotedblleft Parando Serviço\textquotedblright.
%         \item Se o parâmetro for \textit{restart}, deve imprimir \textquotedblleft Reiniciando Serviço\textquotedblright.
%         \item Se o parâmetro for qualquer outra coisa,  deve imprimir \textquotedblleft Uso: servico.sh (start$|$stop$|$restart) \textquotedblright.
% 
%      \end{itemize}
%   \end{frame}

\end{document}
